\documentclass[12pt]{article}
\usepackage{fancyhdr}
\pagestyle{fancy}
\lhead{Enzo Barrett}
\rhead{10-2-18}
\chead{Histography}
\begin{document}
\begin{enumerate}
  \item I think that the first historian, Fay, had the most convincing thesis. Her thesis is far more broad than the other theses which all have particular counties and industries to blame for the war. Fay's is the closest to what happened because even though certain coutries may have had a big effect, in general it was mostly a cummulation of many factors.
  \item I belive that the war was a cause of aliance systems and a few other factors. After the war repercusions were all put on Germany. This was not the true start of a war. Due to the aliance systems that were built up it allowed there to be a situation were a single spark would set off the entire war and that is what happened. A few other factors also contributed to the war including militar leadership, as John stoessinger stated. This is another strong point for the war because no country was really willing to give up the aliances.
  \item A longer time frame since the time of the conflict can give the historian a better view of the event that happened and allow them to read analysis that was given by other historians and allow time for more information and events to happen. This helps put the writing into a more modern perspective for the reader. Even though time can put things into perspective it does not change peoples views as you can see with Niall Ferguson and John Stoessinger. They both wrote there analysis of the war at the same time and Ferguson blames Germany but Stoessinger blames the leadership of the countries. The only very early analysis was writen by Fay and she has more of a vauge piece of writing because it was written so soon after the conflict.
\end{enumerate}
\end{document}
